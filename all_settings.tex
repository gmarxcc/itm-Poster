\documentclass[a1paper,landscape]{tikzposter} 

\tikzposterlatexaffectionproofon %shows small comment on how the poster was made

% Commands
\newcommand{\bs}{\textbackslash}   % backslash
\newcommand{\cmd}[1]{{\bf \color{red}#1}}   % highlights command


% -- PREDEFINED THEMES ---------------------- %

% Backgrounds: Default, Empty, Rays, VerticalGradation, BottomVerticalGradation
% Titles: Default, Basic, Empty, Filled, Envelope, Wave, VerticalShading
% Blocks: Default, Basic, Minimal, Envelope, Corner, Slide, TornOut
% Notes: Default, Sticky, Corner, VerticalShading

\usetheme{$themevar}
$colorvar
$backgroundvar
$titlevar
$blockvar
$notevar

% Title, Author, Institute
\title{$messagevar}
\author{$submessagevar}
\institute{}

\begin{document}

% Title block with title, author, logo, etc.
\maketitle

\block{Block with title}{Content\\~\\~

  % \vspace{1cm}
  % \innerblock{Theorem}{like environment}

  % \vspace{1cm}
  % \innerblock{}{inner block without title}

}

\block{}{~\\
  Block without title\\
  ~\\
}

\note[targetoffsety=-2cm, width=20cm]{
  Note\\ ~\\ ~\\}

\block[titleoffsety=-2cm,bodyoffsety=-2cm]%
{If you like this setting, use the following code:}{
  \bs documentclass\{tikzposter\} \\
  \\
  $themestylevar$colorstylevar$backgroundstylevar$titlestylevar$blockstylevar$notestylevar\\
  \bs title\{Title\}
  \bs author\{Author\}
  \bs institute\{Institute\}\\ 
  \\
  \bs begin\{document\}\\
  \\
  \bs maketitle\\
  \\
  \bs block\{Block\}\{Content\}\\
  \bs end\{document\}
}

\end{document}

