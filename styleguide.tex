\documentclass[9pt]{beamer}

\usepackage{tikz}


\usetheme{Hannover}
\usecolortheme{dove}
\setbeamersize{text margin left=0.2cm,text margin right=0.2cm}
\setbeamertemplate{navigation symbols}{}
\setbeamertemplate{frametitle}[default][left]
\setbeamercolor{frametitle}{fg=red!80!black, bg=gray!20}

% Commands
\newcommand{\bs}{\textbackslash}   % backslash
\newcommand{\cmd}[1]{{\bf \color{red}#1}}   % highlights command


\title{Theme and Style Guide to TikZposter}

\author[] {Elena Botoeva, Richard Barnard, Pascal Richter and Dirk Surmann}


\begin{document}

%%%%%%%%%%%%%%%%%%%%%%%%%%%%%%%%%%%%%%%%%%%% 
\begin{frame}
  \titlepage
\end{frame}

\section{Default}

\begin{frame}
  \frametitle{Your first TikZposter}
  
  \begin{columns}
    \column{.45\textwidth}
    \includegraphics[width=5cm]{all_themes.pdf}

    %%
   \column{.55\textwidth} 
    Making posters in Latex is easy! To see this document, compile the following
    code:

    \medskip
    \hspace{0.2cm}
    \begin{minipage}{.55\textwidth}\small
      \bs documentclass\{tikzposter\}\\
      \\
      \bs title\{Title\}\\
      \bs author\{Author\}\\
      \bs institute\{Institute\}\\
      \\
      \bs begin\{document\}\\
      \\
      \bs maketitle\\
      \\
      \bs block\{Block\}\{Content\}\\
      \bs end\{document\}
    \end{minipage}

  \end{columns}
\end{frame}

\section{Customizing}

\begin{frame}
  \frametitle{Customization of TikZposter}
  
  TikZposter is highly customizable!  There are a number of options
  that can be specified on different levels and define the layout and
  the appearance of your poster.

  \medskip%
  Here we will only discuss customization from the stylistic point of
  view. The goal of this document is to introduce you to the available
  themes and styles that are listed below.

  \begin{description}\small
  \item[Themes:] Default, Rays, Basic, Simple, Envelope, Wave,
    Board, Autumn, Desert.
  
  \item[Color palettes:] Default, BlueGrayOrange, GreenGrayViolet,
    PurpleGrayBlue, BrownBlueOrange.

  \item[Color styles:] Default, Australia, Britain, Sweden, Spain,
    Russia, Denmark, Germany.

  \item[Backgrounds:] Default, VerticalGradation, Rays,
    BottomVerticalGradation, Empty.

  \item[Titles:] Default, Basic, Empty, Filled, Envelope, Wave, VerticalShading.

  \item[Blocks:] Default, Basic, Minimal, Envelope, Corner,
    Slide, TornOut.

  \item[Inner blocks:] Default and Table, along with copies of
    the styles for blocks.

  \item[Notes:] Default, VerticalShading, Corner, Sticky.
  \end{description}
\end{frame}

\subsection{Theme}

\begin{frame}
  \frametitle{Customization of TikZposter: Theme}
  
  \emph{Theme} is a complete collection of all possible options for you poster.

  \medskip
  \begin{columns}[c]
    \column{.45\textwidth}
    \includegraphics[width=5cm, page=3]{all_themes.pdf}

    %%
    \column{.55\textwidth} 
    By \bs usetheme command, you can change the theme of the poster.
    
    \medskip
    \hspace{0.2cm}
    \begin{minipage}{.6\textwidth}\small
      \bs documentclass\{tikzposter\}
    
      \bs usetheme\{Basic\}\\
      \\
      \bs title\{Title\}\\
      \bs author\{Author\}\\
      \bs institute\{Institute\}\\
      \\
      \bs begin\{document\}\\
      \\
      \bs maketitle\\
      \\
      \bs block\{Block\}\{Content\}\\
      \bs end\{document\}
    \end{minipage}

  \end{columns}
\end{frame}

\begin{frame}
  \frametitle{All Themes}
  
  \small\vspace{-0.05cm}
  \begin{tabular}[t]{@{}p{3.5cm}@{~~~}p{3.5cm}@{~~~}p{3.5cm}}
    Default & Rays & Basic\\[-0.03cm]
    \includegraphics[page=1,width=3.5cm]{all_themes.pdf} &
    \includegraphics[page=2, width=3.5cm]{all_themes.pdf} &
    \includegraphics[page=3, width=3.5cm]{all_themes.pdf} \\ [0.03cm]
    Simple & Envelope & Wave\\[-0.03cm]
    \includegraphics[page=4,width=3.5cm]{all_themes.pdf} &
    \includegraphics[page=5, width=3.5cm]{all_themes.pdf} &
    \includegraphics[page=6, width=3.5cm]{all_themes.pdf} \\ [0.03cm]
    Board & Autumn & Desert\\[-0.03cm]
    \includegraphics[page=7,width=3.5cm]{all_themes.pdf} &
    \includegraphics[page=8, width=3.5cm]{all_themes.pdf} &
    \includegraphics[page=9, width=3.5cm]{all_themes.pdf} \\
 \end{tabular}
\end{frame}

\foreach \theme/\i in {%
  Default/1, Rays/2, Basic/3, Simple/4, Envelope/5, Wave/6, Board/7, Autumn/8,
  Desert/9}{
\begin{frame}
  \frametitle{\theme ~Theme}
    \includegraphics[page=\i, width=11.2cm]{all_themes.pdf} 
\end{frame}
}

\subsection{Color}

\begin{frame}
  \frametitle{Customization of TikZposter: Color styles and palettes}
  
  \emph{Color palettes} define the main colors used by the elements of
  poster, while \emph{Color styles} specify the exact colors used in
  depicting the elements of poster, e.g., the color of the text in
  block titles, the frame color of the title, etc.

  \medskip
  \begin{columns}[c]
    \column{.45\textwidth}
    \includegraphics[width=5cm, page=2]{all_colors.pdf}

    %%
    \column{.55\textwidth} %
    By \bs usecolorstyle command, you can change the coloring scheme
    or the main colors, so called color palette, of the poster.

    \medskip
    \hspace{0.2cm}
    \begin{minipage}{.6\textwidth}\small
      \bs documentclass\{tikzposter\}
    
      \bs usecolorstyle[colorPalette=GreenGrayViolet]\\
      \mbox{\qquad\qquad}\{Australia\}\\
      \\
      \bs title\{Title\}\\
      \bs author\{Author\}\\
      \bs institute\{Institute\}\\
      \\
      \bs begin\{document\}\\
      \\
      \bs maketitle\\
      \\
      \bs block\{Block\}\{Content\}\\
      \bs end\{document\}
    \end{minipage}

  \end{columns}
\end{frame}

\begin{frame}
  \frametitle{All Color Styles}
  
  \small\vspace{-0.05cm}
  \begin{tabular}[t]{@{}p{3.5cm}@{~~~}p{3.5cm}@{~~~}p{3.5cm}}
    Default & Australia & Britain\\[-0.03cm]
    \includegraphics[page=1, width=3.5cm]{all_colors.pdf} &
    \includegraphics[page=2, width=3.5cm]{all_colors.pdf} &
    \includegraphics[page=3, width=3.5cm]{all_colors.pdf} \\ [0.03cm]
    Sweden & Spain & Russia\\[-0.03cm]
    \includegraphics[page=4, width=3.5cm]{all_colors.pdf} &
    \includegraphics[page=5, width=3.5cm]{all_colors.pdf} &
    \includegraphics[page=6, width=3.5cm]{all_colors.pdf} \\ [0.03cm]
    Denmark & Germany & \\[-0.03cm]
    \includegraphics[page=7, width=3.5cm]{all_colors.pdf} &
    \includegraphics[page=8, width=3.5cm]{all_colors.pdf} 
  \end{tabular}
\end{frame}

\foreach \col/\i in {%
  Default/1, Australia/2, Britain/3, Sweden/4, Spain/5, Russia/6, Denmark/7, Germany/8}{
\begin{frame}
  \frametitle{\col ~Color Style}
    \includegraphics[page=\i, width=11.2cm]{all_colors.pdf} 
\end{frame}
}

\begin{frame}
  \frametitle{All Color Palettes}
  
  \small
  \begin{tabular}[t]{@{}p{3.5cm}@{~~~}p{3.5cm}@{~~~}p{3.5cm}}
    Default & BlueGrayOrange & GreenGrayViolet\\[-0.03cm]
    \includegraphics[page=1, width=3.5cm]{all_palettes.pdf} &
    \includegraphics[page=2, width=3.5cm]{all_palettes.pdf} &
    \includegraphics[page=3, width=3.5cm]{all_palettes.pdf} \\ [0.05cm]
    PurpleGrayBlue & BrownBlueOrange & \\[-0.03cm]
    \includegraphics[page=4, width=3.5cm]{all_palettes.pdf} &
    \includegraphics[page=5, width=3.5cm]{all_palettes.pdf} &
  \end{tabular}
\end{frame}

\foreach \col/\i in {%
  Default/1, BlueGrayOrange/2, GreenGrayViolet/3, PurpleGrayBlue/4, BrownBlueOrange/5}{
\begin{frame}
  \frametitle{\col ~Color Palette}
    \includegraphics[page=\i, width=11.2cm]{all_palettes.pdf} 
\end{frame}
}

\subsection{Background}

\begin{frame}
  \frametitle{Customization of TikZposter: Background}
  
  \begin{columns}[c]
    \column{.45\textwidth}
    \includegraphics[width=5cm,page=3]{all_backgrounds.pdf}

    %% 
    \column{.55\textwidth} 
    By \bs usebackgroundstyle command, you can change the background of the poster.

    \medskip
    \hspace{0.2cm}
    \begin{minipage}{.6\textwidth}\small
      \bs documentclass\{tikzposter\}
      
      \bs usebackgroundstyle\{Rays\}\\
      \\
      \bs title\{Title\}\\
      \bs author\{Author\}\\
      \bs institute\{Institute\}\\
      \\
      \bs begin\{document\}\\
      \\
      \bs maketitle\\
      \\
      \bs block\{Block\}\{Content\}\\
      \bs end\{document\}
    \end{minipage}

  \end{columns}
\end{frame}

\begin{frame}
  \frametitle{All Background Styles}
  
  \small
  \begin{tabular}[t]{@{}p{3.5cm}@{~~~}p{3.5cm}@{~~~}p{3.5cm}}
    Default & VerticalGradation & Rays\\[-0.03cm]
    \includegraphics[page=1, width=3.5cm]{all_backgrounds.pdf} &
    \includegraphics[page=2, width=3.5cm]{all_backgrounds.pdf} &
    \includegraphics[page=3, width=3.5cm]{all_backgrounds.pdf} \\ [0.05cm]
    BottomVerticalGradation & Empty & \\[-0.03cm]
    \includegraphics[page=4, width=3.5cm]{all_backgrounds.pdf} &
    \includegraphics[page=5, width=3.5cm]{all_backgrounds.pdf} &
  \end{tabular}
\end{frame}

\foreach \back/\i in {%
  Default/1, VerticalGradation/2, Rays/3, BottomVerticalGradation/4, Empty/5}{
\begin{frame}
  \frametitle{\back ~Background}
    \includegraphics[page=\i, width=11.2cm]{all_backgrounds.pdf} 
\end{frame}
}

\subsection{Title}

\begin{frame}
  \frametitle{Customization of TikZposter: Title node}
  
  \begin{columns}[c]
    \column{.45\textwidth}
    \includegraphics[width=5cm, page=6]{all_titles.pdf}

    %%
   \column{.55\textwidth} 
    By \bs usetitlestyle command, you can change the way the title is depicted.

    \medskip
    \hspace{0.2cm}
    \begin{minipage}{.6\textwidth}\small
      \bs documentclass\{tikzposter\}
    
      \bs usetitlestyle\{Wave\}\\
      \\
      \bs title\{Title\}\\
      \bs author\{Author\}\\
      \bs institute\{Institute\}\\
      \\
      \bs begin\{document\}\\
      \\
      \bs maketitle\\
      \\
      \bs block\{Block\}\{Content\}\\
      \bs end\{document\}
    \end{minipage}
  \end{columns}
\end{frame}

\begin{frame}
  \frametitle{All Title Styles}
  
  \small\vspace{-0.05cm}
  \begin{tabular}[t]{@{}p{3.5cm}@{~~~}p{3.5cm}@{~~~}p{3.5cm}}
    Default & Basic & Empty\\[-0.03cm]
    \includegraphics[page=1, width=3.5cm]{all_titles.pdf} &
    \includegraphics[page=2, width=3.5cm]{all_titles.pdf} &
    \includegraphics[page=3, width=3.5cm]{all_titles.pdf} \\ [0.03cm]
    Filled & Envelope & Wave\\[-0.03cm]
    \includegraphics[page=4, width=3.5cm]{all_titles.pdf} &
    \includegraphics[page=5, width=3.5cm]{all_titles.pdf} &
    \includegraphics[page=6, width=3.5cm]{all_titles.pdf} \\ [0.03cm]
    VerticalShading & & \\[-0.03cm]
    \includegraphics[page=7, width=3.5cm]{all_titles.pdf} &
  \end{tabular}
\end{frame}

\foreach \back/\i in {%
  Default/1, Basic/2, Empty/3, Filled/4, Envelope/5, Wave/6, VerticalShading/7}{
\begin{frame}
  \frametitle{\back ~Title}
    \includegraphics[page=\i, width=11.2cm]{all_titles.pdf} 
\end{frame}
}

\subsection{Block}

\begin{frame}
  \frametitle{Customization of TikZposter: Block nodes}
  
  \begin{columns}[c]
    \column{.45\textwidth}
    \includegraphics[width=5cm, page=6]{all_blocks.pdf}

    %%
   \column{.55\textwidth} 
    By \bs useblockstyle command, you can change the style of the blocks.

    \medskip
    \hspace{0.2cm}
    \begin{minipage}{.6\textwidth}\small
      \bs documentclass\{tikzposter\}
    
      \bs useblockstyle\{Slide\}\\
      \\
      \bs title\{Title\}\\
      \bs author\{Author\}\\
      \bs institute\{Institute\}\\
      \\
      \bs begin\{document\}\\
      \\
      \bs maketitle\\
      \\
      \bs block\{Block\}\{Content\}\\
      \bs end\{document\}
    \end{minipage}
  \end{columns}
\end{frame}

\begin{frame}
  \frametitle{All Block Styles}
  
  \small\vspace{-0.05cm}
  \begin{tabular}[t]{@{}p{3.5cm}@{~~~}p{3.5cm}@{~~~}p{3.5cm}}
    Default & Basic & Minimal\\[-0.03cm]
    \includegraphics[page=1, width=3.5cm]{all_blocks.pdf} &
    \includegraphics[page=2, width=3.5cm]{all_blocks.pdf} &
    \includegraphics[page=3, width=3.5cm]{all_blocks.pdf} \\ [0.03cm]
    Envelope & Corner & Slide\\[-0.03cm]
    \includegraphics[page=4, width=3.5cm]{all_blocks.pdf} &
    \includegraphics[page=5, width=3.5cm]{all_blocks.pdf} &
    \includegraphics[page=6, width=3.5cm]{all_blocks.pdf} \\ [0.03cm]
    TornOut & & \\[-0.03cm]
    \includegraphics[page=7, width=3.5cm]{all_blocks.pdf} &
  \end{tabular}
\end{frame}

\foreach \back/\i in {%
  Default/1, Basic/2, Minimal/3, Envelope/4, Corner/5, Slide/6, TornOut/7}{
\begin{frame}
  \frametitle{\back ~Block}
    \includegraphics[page=\i, width=11.2cm]{all_blocks.pdf} 
\end{frame}
}

\subsection{Note}

\begin{frame}
  \frametitle{Customization of TikZposter: Notes}
  
  \begin{columns}[c]
    \column{.45\textwidth}
    \includegraphics[width=5cm, page=4]{all_notes.pdf}

    %%
   \column{.55\textwidth} 
    By \bs usenotestyle command, you can change the style of the notes.

    \medskip
    \hspace{0.2cm}
    \begin{minipage}{.6\textwidth}\small
      \bs documentclass\{tikzposter\}
    
      \bs usenotestyle\{Sticky\}\\
      \\
      \bs title\{Title\}\\
      \bs author\{Author\}\\
      \bs institute\{Institute\}\\
      \\
      \bs begin\{document\}\\
      \\
      \bs maketitle\\
      \\
      \bs block\{Block\}\{Content\}\\
      \bs end\{document\}
    \end{minipage}
  \end{columns}
\end{frame}

\begin{frame}
  \frametitle{All Note Styles}
  
  \small
  \begin{tabular}[t]{p{4.5cm}p{4.5cm}}
    Default & VerticalShading\\
    \includegraphics[page=1, width=4.5cm]{all_notes.pdf} &
    \includegraphics[page=2, width=4.5cm]{all_notes.pdf}\\
    Corner & Sticky\\
    \includegraphics[page=3, width=4.5cm]{all_notes.pdf} &
    \includegraphics[page=4, width=4.5cm]{all_notes.pdf} \\
  \end{tabular}
\end{frame}

\foreach \back/\i in {%
  Default/1, VerticalShading/2, Corner/3, Sticky/4}{
\begin{frame}
  \frametitle{\back ~Block}
    \includegraphics[page=\i, width=11.2cm]{all_notes.pdf} 
\end{frame}
}

\subsection{New Look}

\begin{frame}
  \frametitle{Customization of TikZposter: Your Own Look}

  \footnotesize
  You can completely change the appearance of your poster. Below we
  define a new color style to be used together with the predefined
  background (Default), title (Wave), blocks (Minimal) and notes
  (Default) styles.

  \medskip\scriptsize
  \begin{columns}[t]
    \column{.4\textwidth}
    \begin{minipage}[t]{0.5\textwidth}
      \bs documentclass\{tikzposter\}

      \bs definecolor\{mygray\}\{HTML\}\{CCCCCC\}

      \bs definecolorstyle\{myColorStyle\}\{\\
      \mbox{~~}\bs colorlet\{colorOne\}\{black\}\\
      \mbox{~~}\bs colorlet\{colorTwo\}\{mygray\}\\
      \mbox{~~}\bs colorlet\{colorThree\}\{mygray\}\\
      \}\{\\
      \mbox{~~}\% Background Colors\\
      \mbox{~~}\bs colorlet\{backgroundcolor\}\{colorTwo!50\}\\
      \mbox{~~}\bs colorlet\{framecolor\}\{colorTwo!50\}\\
      \mbox{~~}\% Title Colors\\
      \mbox{~~}\bs colorlet\{titlefgcolor\}\{white\}\\
      \mbox{~~}\bs colorlet\{titlebgcolor\}\{colorOne\}\\
      \mbox{~~}\% Block Colors\\
      \mbox{~~}\bs colorlet\{blocktitlebgcolor\}\{colorTwo!50\}\\
      \mbox{~~}\bs colorlet\{blocktitlefgcolor\}\{black\}\\
      \mbox{~~}\bs colorlet\{blockbodybgcolor\}\{colorTwo!50\}\\
      \mbox{~~}\bs colorlet\{blockbodyfgcolor\}\{black\}\\
      \mbox{~~}\% Innerblock Colors\\
      \mbox{~~}\bs colorlet\{innerblocktitlebgcolor\}\{white\}\\
      \mbox{~~}\bs colorlet\{innerblocktitlefgcolor\}\{black\}\\
      \mbox{~~}\bs colorlet\{innerblockbodybgcolor\}\{white\}\\
      \mbox{~~}\bs colorlet\{innerblockbodyfgcolor\}\{black\}\\
      \mbox{~~}\% Note colors\\
      \mbox{~~}\bs colorlet\{notefgcolor\}\{black\}\\
      \mbox{~~}\bs colorlet\{notebgcolor\}\{white\}\\
      \mbox{~~}\bs colorlet\{notefrcolor\}\{white\}\\
      \}
    \end{minipage}

    %% 
    \column{.6\textwidth} 
    \begin{minipage}[t]{0.5\textwidth}
      \bs usecolorstyle\{myColorStyle\}\\
      % \bs usebackgroundstyle\{Default\}\\
      \bs usetitlestyle\{Wave\}\\
      \bs useblockstyle\{Minimal\}\\
      \\
      \bs usepackage\{avant\}\\
      \bs renewcommand*\bs familydefault\{\bs sfdefault\}\\
      \bs usepackage[T1]\{fontenc\}\\
      \\
      \bs title\{Title\}%
      \bs author\{Author\}%
      \bs institute\{Institute\}\\ 
      \\
      \bs begin\{document\}\\
      \bs maketitle\\
      \bs begin\{columns\}\\
      \mbox{~~}\bs column\{0.5\}\\
      \mbox{~~}\bs block\{Block 1\}\{\\
      \mbox{~~~~}\bs coloredbox\{\\
      \mbox{~~~~~~}Content 1\\
      \mbox{~~}\}\}\\
      \mbox{~~}\bs column\{0.5\}\\
      \mbox{~~}\bs block\{Block 2\}\{\\
      \mbox{~~~~}\bs coloredbox\{\\
      \mbox{~~~~~~}Content 2\\
      \mbox{~~}\}\}\\
      \bs end\{columns\}\\
      \bs end\{document\}
    \end{minipage}
  \end{columns}

  \vspace{-4cm}\hspace{7cm}\includegraphics[width=4cm]{mytheme.pdf}

  ~\\~\\~\\~\\~\\
\end{frame}

\begin{frame}
  \frametitle{My Customized Poster}
    \includegraphics[width=11.2cm]{mytheme.pdf} 
\end{frame}


\end{document}



